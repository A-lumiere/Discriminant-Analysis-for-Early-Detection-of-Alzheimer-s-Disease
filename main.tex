\documentclass[12pt]{article}
\usepackage[utf8]{inputenc}
\usepackage{graphicx}
\usepackage[letterpaper, top = 2 in, right = 1.2in, left = 1.2 in, bottom = 1 in,  headheight=95pt, headsep=30pt]{geometry}
\usepackage{setspace}
\usepackage{titlesec}
\usepackage{fancyhdr}
\usepackage{caption}
\usepackage{tikz}
\usepackage{graphicx}
\usepackage{setspace}
\usepackage{tabularx}
\usepackage{eso-pic}
\usepackage{xcolor}
\usepackage{amsmath}
\usepackage{amssymb}
\usepackage[backend = biber, style = apa]{biblatex}
\addbibresource{references.bib}

% Customization
% Document Customization
\setlength{\parindent}{0.5 in}
\titleformat{\section}[block]{\doublespacing\bfseries\centering}{}{0pt}{}
\titleformat{\subsection}[block]{\doublespacing\bfseries}{}{0pt}{}
\titleformat{\subsubsection}[block]{\doublespacing\bfseries}{}{0pt}{}
\titleformat{\subsubsubsection}[block]{\doublespacing\bfseries}{}{0pt}{}

\newcommand{\verticallineinmargin}{
    \AddToShipoutPictureBG*{
        \AtPageLowerLeft{
            % Adjust the positions and lengths of the lines here
            \put(72,0){\line(0,1){842}}  % Left margin line
            \put(540,0){\line(0,1){842}} % Right margin line
        }
    }
}

\newcommand{\horizontallinemargin}{
    \AddToShipoutPictureBG*{
        \AtPageLowerLeft{
            \put(0, 60){\line(1, 0){700}}
            \put(0, 710){\line(1,0){700}}
            \put(0, 685){\line(1,0){700}}
            \put(0, 35){\line(1,0){700}}
        }
    }
}

% Header and Footer Customization
\pagestyle{fancy}
\fancyhf{}
\fancyhead[C]{
    \verticallineinmargin
    \horizontallinemargin
    \protect\begin{minipage}{\textwidth}
        \centering
        \includegraphics[width = 50pt]{PUPLogo.png} \\
        \vspace*{0.6 cm}
        \textbf{POLYTECHNIC UNIVERSITY OF THE PHILIPPINES}
    \end{minipage}
}
\fancyfoot[C]{\thepage}
\renewcommand{\headrulewidth}{0pt}

\begin{document}
\begin{titlepage}
    \thispagestyle{fancy}
    \fancyfoot[C]{\textcolor{white}{\thepage}}
    \centering
    \vspace*{2cm}  % Adjust vertical space from the top
    \textbf{Linear and Quadratic Discriminant Analysis of Alzheimer's and Non-Alzheimer's Patients
    Based on Lifestyle Factors, Clinical Measurements, and Cognitive and Functional Assessments}
    
    \vspace*{4cm} 
    \normalsize
    In partial fulfillment of the requirements in \\ 
    STAT 20253: Multivariate Analysis

    \vspace*{3cm}
    By \\ [0.5 cm]
    Almer John Sta. Ines \\ [0.3 cm]
    Dencie Mae Saguano \\ [0.3 cm]
    Jerolle Nonato \\ [0.3 cm]
    Joy Ellen Mae Yangyang \\ [0.3 cm]
    Juan Raphael Pimentel \\ [0.3 cm]
    Roldan Libay Jr. 
    \vfill
    \textbf{2025}
\end{titlepage}

% Table of Contents
\newgeometry{top=2in, bottom=1in, left=1.2in, right=1.2in}
\renewcommand{\contentsname}{TABLE OF CONTENTS}
\pagenumbering{roman}
\doublespacing
\tableofcontents

% Chapter 1
\restoregeometry

\pagenumbering{arabic}
\setcounter{page}{1}

\newpage

\section{CHAPTER I \\ THE PROBLEM AND ITS SETTING}
\doublespacing
\noindent

This chapter contains the introduction, statements of the problem, research hypothesis, 
significance of the study, scope and limitations, and definition of terms. 

% Introduction
\subsection{Introduction}
\noindent

Every three seconds, someone in the world develops dementia, adding to the more than 55 million people already 
living with the condtion as of 2020 (\cite{alzint_dementia_statistics}). This number is expected to nearly double 
every 20 years, reaching 78 million in 2030 and a staggering 139 million by 2050. Alzheimer's disease (AD) the most
common cause of dementia, is a progressive brain disorder that gradually affects memory, thinking skills, and the ability
to carry out everyday tasks (\cite{AlzheimersAssociation2021}). As the prevalence of AD continues to rise, it affects people not only altering their lives 
but also placing an emotional and financial strain on families and healthcare system. Early detection is crucial as it allows for 
timely interventions, potential treatments, and better supportfor both patients and caregivers (\cite{Dubois2016}). However, diagnosing AD in its early stages remains challenging, 
as its symptoms often overlap with other cognitive conditions, making it difficult to distinguish from normal aging or mild cognitive impairment (\cite{Jack2018}). Understanding the key
indicators of AD can help improve early diagnosis and lead to better patient outcomes.

A study by Livingston et al. (2020) suggests that lifestyle factors, medical history, clinical measurements, and cognitive and functional assessments can provide essential insights into the
early idenitifaction of Alzheimer's disease. Lifestyle factors, such as Body Mass Index (BMI), smoking status, alcohol consumption, physical activity, diet quality, and sleep quality can impact 
the risk of developing AD (\cite{nih_healthy_lifestyle}). Additionally, certian medical conditions, including hypertension, diabetes, cardiovascular, diseases, depression, family history of AD, and
head injury have been linked to increased susceptability to AD (\cite{Kivipelto2018}). Clinical measurements, particularly cardiovascular health indicators, have been increasingly recognized as significant
risk factors for the disease. Moreover, a study conducted by Leszek et al. (2020) suggests that cardiovascular diseases can contribute to the development and progression of AD by impairing blood flow to the brain
and causing inflammation (\cite{Leszek2021}). Functional assessments, which evaluate how well a person can manage daily tasks and adjust to cognitive changes, can also be an important tool in detecting early signs of AD. (\cite{sperling2011})

The effectiveness of classification models in predicting AD is crucial for developing reliable diagnostic tools. LDA and QDA are two widely used statistical techniques for classification problems. LDA assumes equal covariance structures
among groups, making it more effective when the assumption holds, while QDA relaxes this assumption, allowing for more flexibility in classification (\cite{elements_statistical_learning})

The study aims to investigate significant differences between Alzheimer's patients and Non-Alzheimer's patients concerning lifestyle factor, clinical measurements, and cognitive and functional assessments. Additionally, it seeks to identify 
substantial indicators of AD and compare the discriminative power of LDA and QDA in early prediction. By analyzing there aspects, this research contributes to the development of more accurate and efficient classification models for early AD 
diagnosis. 

\subsection{Objectives}
\noindent

The general objective of the study is to accurately discriminate Alzheimer's patients to Non-Alzheimer's patients based on their lifestyle factors, clinical measurements, and cognitive and functional assessment.

Specifically, it aims to achieve the following objectives. 

\begin{enumerate}
    \item To identify which clinical indicators can effectively discriminate Alzheimer's patients to Non-Alzheimer's patients. 
    \item To optimize a discriminant classification model through feature reduction. 
    \item To compare the discriminative power of LDA and QDA for an early prediction of AD. 
\end{enumerate}

\subsection{Statements of the Problem}
\noindent

Following the mentioned objectives, the study sought to answer the following questions. 
\begin{enumerate}
    \item Are there significant differences between Alzheimer's and Non-Alzheimer's patients in terms of:
    \begin{enumerate}
        \item Lifestyle Factors
        \item Clinical Measurements
        \item Cognitive and Functional Assessment
    \end{enumerate}
    \item What are the substantial indicators of AD? in terms of: 
    \begin{enumerate}
        \item Lifestyle Factors
        \item Clinical Measurements
        \item Cognitive and Functional Assessment
    \end{enumerate}
    \item How do the discriminative powers of LDA and QDA compare in the early prediction of AD?
\end{enumerate}

\subsection{Significance of the Study}
\noindent

This research aims to develop a classification model for the early prediction of AD by analyzing key factors such as lifestyle, clinical measures, and cognitive and functional assessments. 
The findings of this study will be significant to the following sectors: 

\textbf{Healthcare Practitioner and Neurologist}. This study will provide a data-drive aproach for early AD prediction, aiding in timely intervention and personalized treatment plan. Furthermore, 
this research will enhance the understanding of the most substantial indicators of AD based on statistical multivariate models. 

\textbf{Researchers and Data Scientist}. The study will contribute to the growing field of medical data analysis by exploring the effectiveness of LDA and QDA in disease membership classification. It 
will serve as a foundation for the future studies integrating machine learning and statistical methods in healthcare analytics which will potentially lead to more advanced diagnostic tools. 

\textbf{Patients and Families}. This study will benefit them as early detection tools can help them prepare for disease management and necessary lifestyle adjustments. This will also raise awareness of the 
key lifestyle and medical factors that may contribute to the risk of AD and encourage them to have preventive measures that may delay the onset of disease.

\textbf{Public Health and Policy Makers}. This study holds significance to the general public health sector by supporting the development of data-drive healthcare policies focused on early screening and intervention
programs. The insights gained from this research can help allocate resources in AD research, prevention and treatment programs.

\subsection{Scopes and Limitations}
\noindent

This study aims on developing a classification model for early prediction of AD using LDA and QDA. It focuses on analyzing and identifying important factors such as lifestyle habits, clinical measurements, and cognitive and 
functional assessments to distinguish between Alzheimer's and Non-Alzheimer's patients. The study also compares the discriminative power of LDA and QDA in identifying substantial predictors of the disease and compares their 
classification performance. 

The dataset used in this study consists of 2,149 patient records with unique identifications ranging from 4571 to 6900. It includes demographic details, lifestyle factors, medical history, clinical measurements, cognitive and functional
assessments, symptoms, and Alzheimer's diagnosis. Since the dataset is synthetically generated and designed from educational purposes, it provides a structured and controlled environment for analysis. However, because it is not based on real patient data, 
it may not fully capture the variability and complexity of real-world medical dataset. 

Despite its strengths, the study has certain limitations. Since the dataset is synthetic and not sourced from actual medical records, the generalizability of the findings to real-worlc clinical settings may be restricted. Additionally, LDA and QDA rely on certain
statistical assumptions, like normality and homoscedasticity for LDA, which may effect their predictive accuracy in more complex datasets. This study also does not compare LDA and QDA with other machine learning models, limiting the analysis to just these two multivariate
techniques. 

\subsection{Defintion of Terms}

\section{CHAPTER II \\ REVIEW OF LITERATURE AND STUDIES}

\subsection{Factors Analyzing Alzheimer's Classification Using Discriminant Analysis}
\subsection {Lifestyle Factors}
\noindent

Lifestyle behaviors play a vital role in maintaining cognitive health and reducing the rish of Alzheimer's disease. Engaging in regular physical activity, for instance, has been linked to a lower likelihood of cognitive decline. \cite{Dominguez2021} emphasized that aerobic
exercises not only boost brain function but can also help delay the onset of dementia. Similarly, following a healthy diet, particularly one rich in fruits, vegetables, and whole grains, like the Mediterranean diet, has been shown to enhance cognitive performance and decrease
the risk of Alzheimer's. Maintaining an active social life also contributes significantly, as frequent social interactions and participation in activities foster cognitive resilience and lower the chanves of developing dementia. (\cite{Dominguez2021})

Sleep quality and duration are emerging as important factors in Alzheimer's risk. Poor sleep habits, such as not getting enough rest or experiencing disrupted sleep, have been linked to a buildup of amylod-beta in the brain, a hallmark of Alzheimer's. \cite{Dominguez2021} highlights
that good-quality sleep is essential for brain health. It helps clear out metabolic waste products and supports memory consolidation, making it a critical part of maintaining cognitive function.

\subsection{Clinical Measurements}
\noindent

Recent advancements in biomarker research have greatly improved the ability to classigy and detect Alzheimer's disease at an early stage. Blood-based biomarkers, such as specific microRNAs (miRNAs), are showing promimse in identifying individuals at risk even before symptoms appear 
(\cite{JAMA2019}). Genetics also play a key role,, with the APOE4 allele being a well-known risk factor. People who carry this variant have a higher chance of developing Alzheimer's, and integrating genetic information into classification models can enhance accuracy in predicting the
disease (\cite{JAMA2019}).

Health conditions like diabeetes, heart disease, and high blood pressure have been linked to a higher risk of developing Alzheimer's disease. These issues can worsen the impact of vascular problems on brain function and may even interact with Alzheimer-related changes in the brain. Taking
steps to manage these conditions through healthy lifestyle choices and medical treatments, play a key role in lowering the overall risk of dementia (\cite{PMC2021}).

\subsection{Cognitive and Functional Assessment}
\noindent

Cognitive tests play a crucial role in identifying and classifying Alzheimer's disease. Standatd assessments like the Mini-Menteal State Examination (MMSE) and the Montreal Cognitive Assessment (MoCA) help measure cognitive function in a structured way. The results from these tests provide 
valuable insights that can improve classification models, making it easier to distinguish between healthy individuals, those with mild cognitive impairment, and those with Alzheimer's disease (\cite{PMC2021}). Functional assessment, on the other hand, focuses on understanding how well a person
can handle everyday tasks and adapt to changes in their thinking abilities. Tools like the Alzheimer's Disease Cooperative Study-Activities of Daily Living Inventory (ADCS-ADL) and the Functional Activities Questionnaire (FAQ) are often used to evaluate these abilities (\cite{Custodio2022}). Supported by
the study of \cite{Cummings2017} which stated that these assessments go beyond cognitive tests, offering a more complete picture of how Alzheimer;s disease affects a person's overall functioning and quality of life.

\subsection{Comparative Analysis of LDA and QDA in Alzheimer's Prediction}
\noindent

Linear Discriminant Analysis (LDA) and Quadratic Discriminant Analysis (QDA) are commonly used classification techniques in medical datasets, particularly for Alzheimer's disease (AD). According to \cite{jain2022}, LDA stands out for its simplicity, especially when data is linearly seperable, while QDA
handles more complex, non-linear patterns effectively but at a higher computational cost. This makes the choice between the two methods dependent on the nature of the dataset.

Recent research emphasizes the strengths of both approaches. LDA is praised for its ease of interpretation and ability to identify key predictors, offering reliable and computationally efficient models. In contrast, QDA, with its ability to handle non-linearities, excels in sensitivity, especially in
detecting early cognitive decline, but may sacrifice some interpretability (\cite{arbabshirani2017}). Studies show that LDA often has higher specificity, whereas QDA is more sensitive in distinguishing between healthy individuals and those with Alzheimer's (\cite{nguyen2020}).

Model interpretability and computational complexity remain critical considerations. LDAs straightforward structure allows for a better understanding of variable relatkionships, making it a practical choice in clinical settings. On the other hand, QDA's flexibility can come at the cost of higher
computational demands and reduced clarity in variable significance (\cite{wang2018}).

\subsection{Linear Discriminant Analysis (LDA)}
\noindent
LDA was used as a classification method to predict Alzheimer's disease patients based on clinical biomarkers, neuroimaging data, and metabolic profiles. This technique has been utilized in several studies to enhance diagnostic accuracy and distinguish between Alzheimer's disease, mild cognitive impairment
(MCI), and healthy controls. Data records mainly consisted of medical imaging data, metabolic biomarkers, and cognitive assessment scores, which were processed through feature selection and dimensionality reduction teechniques to enhance classification performance.

LDA assumes that different classes (AD and non-AD groups) share the same covariance structure and constructss linear boundaries to maximize class separability. In the study by (\cite{le2020_lda_highdimensional}), an adapted LDA approach was implemented to handle high-dimensional medical datasetss, ensuring better feature selection and
classification accuracy. Similarly, \cite{salasgonzalez2010_factor_analysis_lda} utilizied LDA in combination with factor analysis to select the most relevant features from 18F-FDG PET images, optimizing classification between Alzheimer's and control groups.

The study by \cite{yilmaz2021_metabolic_biomarkers_ad} applied artificial intelligence and machine learning techniques alongside discriminant analysis to identify biomarkers associated with Alzheimer's progression. Feature selection was crucial in reducing redundancy and enhancing predictive capability. \cite{maroco2011_data_mining_dementia}
conducted a comparative analysis between LDA, logistic regression, neural networks, support vector machines, classification trees, and random forests. They assessed model performance using key evaluation metrics, demonstrating that LDA achieved high accuracy in distringuishing AD patients from controls when optimized with feature selection.

\subsection{Quadratic Discriminant Analysis (QDA)}
\noindent

QDA, unlike LDA, relaxes the assumption of shared covariance struvtures and allows each class to have its own covariance matric, making it suitable for datasets where feature dsitributions exhibit non-linearity. This flexibility enables QDA to better handle complex biomarker distributions in Alzheimer's disease classification.

QDA has demonstrated strong classification performance in Alzheimer's research. Studies such as \cite{zhang2017_qda_neuroimaging} and \cite{pereira2020_qda_mri} explored the application of QDA in neuroimaging-based Alzheimer's diagnosis, showing that its ability to model class-specific covariance matrices led to improved classification precision
and recall. Additionally, the work of \cite{maroco2011_data_mining_dementia} indicated that QDA often achieved higher F1-scores in datasets with more complex patterns compared to LDA.

Furthermore, studies have shown that QDA performs well in scenarios where biomarker distributions exhibit high variability. For instance, \cite{lee2015_qda_mri_biomarker} demonstrated that QDA outperformed LDA in identifying Alzheimer's subtypes based on MRI-deived biomarkers, particularly in cases with overlapping clinical features. Similarly, 
\cite{garciarodriguez2016_qda_csf} employed QDA to analyze cerebrospinal fluid biomarkers, showing significant improvements in classifying AD and MCI patients. The higher Area Under the Curve (AUC) values reported in these studies suggest that QDA provides superior discrimination between AD and non-AD cases when biomarker variability is high.

\subsection{Feature Selection Techniques in Discriminant Analysis}
\noindent
\textbf{Recursive Feature Elimination (RFE)}
\noindent

RFE is a backward feature elimination technique that iteratively removes the least significant features based on their impact on model accuracy. This study implements RFE to systematically refinea feature subsets, ensuring that only the most releveant biomarkers contribute to classification. The process starts by training an initial model on all features,
ranking their importance, and recursively eliminating the least informative ones until an optimal subset remains.

Previous studies, such as \cite{balakrishnan2020_rfe_ann_ad}, combined RFE with artificial neural netwroks to identify critical features for Alzheimer's disease classification. Similarly, \cite{alshamlan2018_biomarker_gene_selection} demonstrated that RFE facilitates biomarker gene selection, improving classification peformance. In this study, RE is applied
to clninical biomarkers, cognitive scores, and neuroimaging markers to eliminate redundant features, refine the input space, and mitigate overfitting, thereby enhancing the performance of LDA and QDA.

\subsection{Lasso Regression}
\noindent

Lasso regression, a form of regularization, employs an L1 penalty constraint to shrink the coefficients of less relevant features to zero, effectively selecting only the most significant predictors. This study applies Lasso regression to ensure that the discriminant functions of LDA and QDA are derived from the most influential biomarkers while minimizing
noise and improving model generalizability.

\cite{gu2020_feature_selection_ad} evaluated Lasso regression as a feature selection method for Alzheimer's diagnosis, emphasizing its capacity to identify high-impact features while preserving interpretability. Additionally, \cite{spooner2020_ensemble_feature_selection} demonstrated Lasso's efficacy in biomarker discovery, enhancing model rrobustness in
distinguishin between disease stages. by incorporating Lasso regression, this study aims to improve classification accuracy by retaining only the most essential biomarkers.

\section{CHAPTER III \\ METHODOLOGY}

\newpage
\raggedleft
\defbibheading{bibliography}[\refname]{\section*{\raggedleft#1}}
\printbibliography

\end{document}